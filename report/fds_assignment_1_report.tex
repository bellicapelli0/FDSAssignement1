\documentclass{article}
\usepackage{amsmath}
\usepackage{physics}
\usepackage{mathtools}
\usepackage[hidelinks]{hyperref}
\usepackage{verbatim}
\usepackage{graphicx}
\usepackage{caption}
\usepackage{cleveref}
\usepackage{listings}
\usepackage{float}
\usepackage{appendix}
\usepackage[useregional]{datetime2}
%\usepackage{interval}
\usepackage[backend=biber, style=apa]{biblatex}
\addbibresource{cap_bibliography.bib}

\usepackage[margin=1in, left=1.5in, includefoot]{geometry}

\usepackage{fancyhdr}
\pagestyle{fancy}
\fancyhead{}
\fancyfoot{}
\fancyhead[R]{\thepage}
\lhead{\textbf{Assignment Team 12}}

\newcommand\email[1]{
    \href{mailto:#1}{\url{#1}}
}

\usepackage{hyperref}
\hypersetup{
  colorlinks=true,
  allcolors=[rgb]{0.129,0.341,0.769}
}

\setlength{\parindent}{0em}
\begin{document}
\begin{titlepage}
  \newcommand{\HRule}{\rule{\linewidth{0.5mm}}}

    \begin{center}
      %\HRule
      %\\[0.5 cm]
      \rule{\linewidth}{0.5mm} \\
      \vspace{0.5 cm}
      \huge{\bfseries Assignment 1: Studying 2016 US Elections Through
        Analysing Twitter Data}
      \rule{\linewidth}{0.5mm}
      \\[1.2 cm]
      \begin{minipage}{0.45\textwidth}
        \begin{flushleft}
        \large
        António Mendes\\
        {\small\email{17amendes@gmail.com}} \\
        {\small 11925051}
        \end{flushleft}
      \end{minipage}
      ~
      \begin{minipage}{0.4\textwidth}
        \begin{flushright}
          \large
          Judit Győrfi\\
          {\small judit.gyorfi@student.uva.nl}\\
          {\small 13209647}
        \end{flushright}
      \end{minipage}
      \\[1cm]
      \begin{minipage}{0.45\textwidth}
        \begin{flushleft}
        \large
        Orlando Scarpa\\
        {\small \email{orlando.scarpa@student.uva.nl}}\\
        {\small 13266918}
        \end{flushleft}
      \end{minipage}
      ~
      \begin{minipage}{0.4\textwidth}
        \begin{flushright}
          \large
          Horváth Ádam\\
          {\small email address}\\
          {\small UVA-net ID}
        \end{flushright}
      \end{minipage}
      \\[1cm]
      \begin{minipage}{0.5\textwidth}
        \begin{center}
          \large
          Miklos Kosarszky\\
          {\small\email{miklos.kosarszky@student.uva.nl}}\\
          {\small 13242857}
        \end{center}
      \end{minipage}
      \vfill
      \vfill
      \vfill
    \end{center}
  \end{titlepage}

  \tableofcontents
  \thispagestyle{empty}
  \pagebreak
  \setcounter{page}{1}
  \setcounter{section}{0}
  \section{Data Preparation}
  After extracting all the json objects and flattening all the fields
  the DataFrame had 35 columns, Of these, 10 were selected as valuable
  insights into the data.\\
  
  Using the columns relating to the language and country code, the
  DataFrame was reduced to only tweets in english and posted from the
  United States. After this, the column containing the full name of
  the location of origin of the tweet was used to extract the two
  letter code of the state of origin, keeping in mind to exclude
  territories whose inhabitants don’t vote for the president like
  Puerto Rico and Guam. Of the 517,724 tweets in english from the
  United States, less than 5000 tweets were excluded in this
  manner. To clean the text and prepare it to be tokenized, all text
  in the tweets was stripped of punctuation, special characters, new
  lines, hyperlinks and trailing spaces. Furthermore, stop words were
  removed and select hashtags and mentions were counted to give a
  rudimentary estimate of the candidate being talked about in each
  tweet.
  
  \section{Data Splitting}
  \section{Sentiment Analysis}
  \section{Topic Modelling}
  \section{Results Understanding}
  \section{Limitations}
  \section{Context}
\end{document}
%%% Local Variables:
%%% mode: latex
%%% TeX-master: t
%%% End:
