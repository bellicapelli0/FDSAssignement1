\documentclass{article}
\usepackage{amsmath}
\usepackage{physics}
\usepackage{mathtools}
\usepackage[hidelinks]{hyperref}
\usepackage{verbatim}
\usepackage{graphicx}
\usepackage{caption}
\usepackage{cleveref}
\usepackage{listings}
\usepackage{float}
\usepackage{appendix}
\usepackage[useregional]{datetime2}
%\usepackage{interval}
\usepackage[backend=biber, style=apa]{biblatex}
\addbibresource{cap_bibliography.bib}

\usepackage[margin=1in, left=1.5in, includefoot]{geometry}

\usepackage{fancyhdr}
\pagestyle{fancy}
\fancyhead{}
\fancyfoot{}
\fancyhead[R]{\thepage}
\lhead{\textbf{Assignment Team 12}}

\newcommand\email[1]{
    \href{mailto:#1}{\url{#1}}
}

\usepackage{hyperref}
\hypersetup{
  colorlinks=true,
  allcolors=[rgb]{0.129,0.341,0.769}
}

\setlength{\parindent}{0em}
\begin{document}
\begin{titlepage}
  \newcommand{\HRule}{\rule{\linewidth{0.5mm}}}

    \begin{center}
      %\HRule
      %\\[0.5 cm]
      \rule{\linewidth}{0.5mm} \\
      \vspace{0.5 cm}
      \huge{\bfseries Assignment 1: Studying 2016 US Elections Through
        Analysing Twitter Data}
      \rule{\linewidth}{0.5mm}
      \\[1.2 cm]
      \begin{minipage}{0.45\textwidth}
        \begin{flushleft}
        \large
        António Mendes\\
        {\small\email{17amendes@gmail.com}} \\
        {\small 11925051}
        \end{flushleft}
      \end{minipage}
      ~
      \begin{minipage}{0.4\textwidth}
        \begin{flushright}
          \large
          Judit Győrfi\\
          {\small \email{judit.gyorfi@student.uva.nl}}\\
          {\small 13209647}
        \end{flushright}
      \end{minipage}
      \\[1cm]
      \begin{minipage}{0.45\textwidth}
        \begin{flushleft}
        \large
        Orlando Scarpa\\
        {\small \email{orlando.scarpa@student.uva.nl}}\\
        {\small 13266918}
        \end{flushleft}
      \end{minipage}
      ~
      \begin{minipage}{0.4\textwidth}
        \begin{flushright}
          \large
          Adam Horvath-Reparszky\\
          {\small \email{adam.horvath-reparszky@student.uva.nl}}\\
          {\small 13326481}
        \end{flushright}
      \end{minipage}
      \\[1cm]
      \begin{minipage}{0.5\textwidth}
        \begin{center}
          \large
          Miklos Kosarszky\\
          {\small\email{miklos.kosarszky@student.uva.nl}}\\
          {\small 13242857}
        \end{center}
      \end{minipage}
      \vfill
      \vfill
      \vfill
    \end{center}
  \end{titlepage}

  \tableofcontents
  \thispagestyle{empty}
  \pagebreak
  \setcounter{page}{1}
  \setcounter{section}{0}
  \section{Data Preparation}
  After extracting all the json objects and flattening all the fields
  the DataFrame had 35 columns, Of these, 10 were selected as valuable
  insights into the data.\\
  
  Using the columns relating to the language and country code, the
  DataFrame was reduced to only tweets in english and posted from the
  United States. After this, the column containing the full name of
  the location of origin of the tweet was used to extract the two
  letter code of the state of origin, keeping in mind to exclude
  territories whose inhabitants don’t vote for the president like
  Puerto Rico and Guam. Of the 517,724 tweets in english from the
  United States, less than 5000 tweets were excluded in this
  manner. To clean the text and prepare it to be tokenized, all text
  in the tweets was stripped of punctuation, special characters, new
  lines, hyperlinks and trailing spaces. Furthermore, stop words were
  removed and select hashtags and mentions were counted to give a
  rudimentary estimate of the candidate being talked about in each
  tweet.
  
  \section{Data Splitting}
  \section{Sentiment Analysis}
  \section{Topic Modelling}
  \section{Polarity}
  The already cleaned data provides us with many possibilities in
  terms of data analysis. One of the main focus of the analysis is on
  sentiment analysis of the Twitter texts. Python has a wide variety
  of Natural Language Processing (NLP) tools that are helpful when
  looking for the sentiment value of a particular text, and in our
  case, tweets. Some of the packages, such as NLTK and TextBlob can
  deal with strings mathematically. To find out the polarity of our
  tweets, we have used the TextBlob tool which classifies the text
  into negative, neutral and positive tweets between the values $-1$ and
  $1$, where negative numbers tend to refer to tweets with negative
  wording and positives refer to tweets with positive words. TextBlob
  also creates a second output called subjectivity, with values
  between $0$ and $1$  where $0$ is an objective tweet and $1$ is a subjective
  one. In this assignment, however, we are only focusing on polarity
  as this can be more correlated to actual vote results.%\\

  The first step in our analysis is to create a new column called
  “polarity” in the already cleaned data with the TextBlob package
  based on the tweet texts. The new column will contain the values of
  the polarities of the particular tweets. Next, we are grouping the
  data by states and taking the average value of the polarities per
  state. As a result ,we get the sorted polarity values for the 50
  states. We can observe that the average polarity values of the
  tweets in general are between $-0.2$ and $0.2$.

  In order to get a better insight on how tweets can be related to
  votes results, we have split up our dataset to tweets which are
  mainly about Trump and to tweets about Hillary. After that, we
  follow the same procedure as before, and group the average polarity
  scores of the two candidates per state. Next, we load an external
  dataset from the $2016$ U.S elections
  (“usa-2016-presidential-election-by-county.csv”) to compare the
  results with the polarity of the tweets. The results state that
  Trump got the most positive comments from states like Montana,
  Wisconsin and Kentucky. These are states where Trump also got more
  votes than Hillary Clinton.  
  
  \begin{table}[H]
    \centering
    \begin{tabular}{||p{2cm} p{3cm} p{3cm} ||}
      \hline
      state & polarity & name \\ \hline
      NE & 0.0819 & Nebraska \\
      MT & 0.0825 & Montana \\
      WI & 0.1080 & Wisconsin \\
      KT & 0.1120 & Kentucky \\
      \hline
    \end{tabular}
    \caption{\label{tab:} }
  \end{table}

  As for Hillary, we can conclude that she received negative tweets in
  states which are Republican bastions such as Idaho, North Dakota and
  Wyoming. Correspondingly, she received positive tweets from states
  such as Washington D.C and New Hampshire which indeed voted for
  Hillary. We have calculated a correlation score between polarity and
  the winner of a state ($1$ : Trump, $0$: Hillary). The resulting
  correlation for tweets containing Trump and winner:Trump  is
  $0.257$. The result states that the more positive the tweet about
  Trump is, the more likely that state has voted for Trump. Vice
  verse, if the tweets are more positive about Hillary, it is less
  likely that people in that state voted for Trump. (Correlation:
  $-0.189$). However, these correlations are not high enough (smaller
  than $0.3$) to say that the polarity of a tweet has a decisive impact
  on the vote results.

  \begin{table}[H]
    \centering
    \begin{tabular}{||p{2cm} p{3cm} p{3cm} ||}
      \hline
      state & polarity & name \\ \hline
      ID & -0.0236 & Idaho \\
      ND & -0.0158 & North Dakota \\
      WY & -0.0148 & Wyoming \\
      UT & -0.0133 & Utah \\
      DC & 0.0786 & District of Columbia \\
      NH & 0.0822 & New Hampshire\\
      \hline
    \end{tabular}
    \caption{\label{tab:} }
  \end{table}

  To get a better overview, we have included a few maps which reflect
  whether the tweets in a state were positive or negative about
  Hillary or Trump. Furthermore, we have also added a map about the
  votes results in 2016 based on the external dataset. For the
  visualisation, we have used plotly and matplotlib.  
  
  \section{Results Understanding}
  \section{Limitations}
  \section{Context}
\end{document}
%%% Local Variables:
%%% mode: latex
%%% TeX-master: t
%%% End:
